\chapter{Formula��o Matem�tica}
\label{ch:form_mat}

Neste cap�tulo ser� apresentada a formula��o matem�tica utilizada para modelar o escoamento de fluidos em meios porosos. As equa��es apresentadas se baseiam na hip�tese de um escoamento tri-f�sico do tipo \emph{black-oil}, onde as fases s�o normalmente �gua, �leo e g�s, sendo que admite-se que n�o haver� troca de massa entre as fases �gua e �leo, e que o componente �leo n�o pode passar para a fase gasosa, entretanto a fase g�s pode existir tanto isoladamente quanto dissolvida na fase �leo.

\section{Engenharia de Reservat�rio de Petr�leo}
\label{sc:reservatorio}
O objetivo final ao se realizar simula��es de reservat�rios de petr�leo � dar suporte as decis�es do engenharia de reservat�rio durante as diversas fases do projeto de um campo de petr�leo.

\section{Equa��o de Press�o}
\label{sc:eq_press}
Considerando a lei da conserva��o de massa, podemos considerar

\section{Equa��o de Satura��o}
\label{sc:eq_sat}
Calculando a velocidade a partir do grandiente de press�o obtido no passo anterior, podemos ent�o definir de modo expl�cito uma equa��o para a satura��o.
\begin{equation} \label{eq:sat}
\phi \frac{\partial S}{\partial t} + f_{,S} \mathbf{u} \cdot \nabla S + \nabla \cdot 
(\mathbf{\kappa}\lambda_{o}f(\rho_{w} - \rho_{o})\mathbf{g})=0
\end{equation}
a partir de (\ref{eq:sat}) podemos derivar...

