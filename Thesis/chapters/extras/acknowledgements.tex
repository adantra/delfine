Ao Prof.~Paulo Lyra, cuja mentoria foi fundamental para a minha forma��o como profissional, pela orienta��o, confian�a e apoio inestim�veis desde os tempos de inicia��o cient�fica. Agrade�o especialmente pela amizade e paci�ncia demonstradas ao longo dos anos.

\vspace{3 mm}

\noindent
Ao Prof.~Ramiro Willmersdorf, pelos coment�rios sempre sucintos e pertinentes, e tamb�m por sua memor�vel palestra plen�ria no CILAMCE 2008, cujas ideias influenciaram bastante o rumo tomado neste trabalho.

\vspace{3 mm}

\noindent
Aos professores �zio Ara�jo e Alessandro Antunes, pela participa��o na banca examinadora e pelos coment�rios para melhoria da disserta��o.

\vspace{3 mm}

\noindent
Ao Prof.~Darlan Carvalho, pelas v�rias sugest�es para este trabalho e principalmente por todos os ensinamentos transmitidos ao longo dos anos.

\vspace{3 mm}

\noindent
Aos colegas de LABCOM (Rafael, Adriano, Rodrigo, H�lder, Danilo, Ana Paula, etc.), pelos momentos de trabalho e de descontra��o.

\vspace{3 mm}

\noindent
� Petrobras, pelo apoio constante ao grupo PADMEC.

\vspace{3 mm}

\noindent
� minha m�e, Neuce, e ao meu pai, Cosme, aos quais serei eternamente grato pelo apoio incondicional em todos os momentos, pelo incentivo � minha forma��o acad�mica e principalmente por serem os meus maiores orientadores gra�as a todos os valores ensinados atrav�s de exemplos di�rios.

\vspace{3 mm}

\noindent
� minha irm�, Monique, pelo carinho, amizade e por ser sempre um exemplo de dedica��o e vontade.

\vspace{3 mm}

\noindent
� minha esposa, Caroline, por seu amor e por ter estado sempre ao meu lado durante esta jornada. Este trabalho � t�o seu quanto meu. \emph{Ich liebe dich!}