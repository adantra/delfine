The simulation of multiphase flows in porous media impose many numerical challenges due to a series of factors as the high anisotropic and heterogeneous media handled in this type of analysis, the coupled elliptic-hyperbolic mathematical nature of the associated Partial Differential Equations (PDEs),  among others. Even after the mathematical and numerical formulations used to model the flow are defined, there is still another challenge regarding the codification of these methods, because it is usually a very time-consuming task to develop computer programs that implement formulations for general and/or complex cases. This master thesis presents the implementation of a software written using the Python programming language and the FEniCS computational tool for the automatic generation of low-level code in C++ applied in the numerical solution of mono- and biphasic flows in porous media using the Finite Element Method (FEM). The classical Galerkin FEM and the Mixed Finite Element Method (MFEM) were tested for the solution of the pressure (pressure and velocity for MFEM) and the Streamline Upwind Petrov Galerkin (SUPG) stabilized FEM with shock capturing operator for the saturation equation. A convergence accelaration technique via Algebraic Multigrid (AMG) was used for the solution of the linear system of equations derived from the Galerkin FEM discretization. The methods described here are general enough to handle with three-dimensional, heterogeneous and anisotropic problems. Examples are shown and results discussed for one- and two-dimensional problems in homogenous and heterougenous domains with iso- and anisotropic permeability tensors. The comparisons of the results obtained in work with those from analytical solutions and literature references show the potential of the developed tool for the simulation of flows in porous media.

\begin{keywords}
Porous Media Flow, Finite Element Method, Automatic Modelling
\end{keywords}
