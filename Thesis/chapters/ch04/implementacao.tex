\chapter{Implementa��o Computacional}
\label{ch:implementacao}

Neste cap�tulo ser�o os apresentado os pacotes FEniCS/Dolfin e PyAMG, o primeiro foi extensivamente utilizado ao longo deste trabalho para a montagem das matrizes de elementos finitos e o segundo foi utilizado para resolver o sistema de equa��es provenientes das matrizes obtidas.

Al�m disso, ser� apresentada a estrutura geral do programa gerado neste trabalho e as principais caracter�sticas do mesmo.

\section{Estrutura Geral do Program}
\label{sc:est_prog}
O programa foi escrito na linguagem python, a qual apresenta v�rias vantagens sobre as outras, dentre as quais podemos citar:

\section{FEniCS/Dolfin}
\label{sc:dolfin}
De modo a automatizar a gera��o de c�digo de baixo-n�vel do problema variacional necess�rio para a implementa��o dos m�todos descritos no cap�tulo \ref{ch:form_num}, foi utilizada a ferramenta FEniCS/Dolfin, a qual consiste em uma s�rie de chamadas (\emph{wrappers}) para rotinas otimizadas escritas em C++. Essa interface pode ser acessada tanto atrav�s da pr�pria linguagem C++ quanto atrav�s do Python, tendo esta �ltima alternativa sido escolhida devido a facilidade e rapidez para desenvolvimento de c�digo com a mesma.

\section{PyAMG}
\label{sc:pyamg}
O PyAMG � uma biblioteca de fun��es escritas em fundamentalmente em Python e com algor�timos chave do ponto de vista de performance escritos em C++

