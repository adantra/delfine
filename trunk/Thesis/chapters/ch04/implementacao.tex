\chapter{Implementa��o Computacional}
\label{ch:implementacao}

All the computational tools described in this work present great challenges not only in terms of theoretical understanding but also in terms of software development, as it is of paramount importance to adequately implement these methods in a way that a positive trade-off between generality and performance is found, as these two goals often lead the coding directives to different directions.

\section{Estrutura Geral do Program}
\label{sc:est_prog}
O programa foi escrito na linguagem python, a qual apresenta v�rias vantagens sobre as outras, dentre as quais podemos citar:

\section{FEniCS/Dolfin}
\label{sc:dolfin}
For the generation of the code responsible for assembling the matrices and vectors necessary for the finite element analysis of the problem we used the \emph{FEniCS/DOLFIN} library \citep{Logg2010}, which allows the developer to write the linear and bilinear forms presented in Eq. (\ref{eq:galerkin}) using a script written in \emph{Python} in a very high-level way, that resembles the mathematical notation,  and this script is then automatically converted in optimized \emph{C++} code, which can run quite effectively for general problems.
% Explicar com MUITO mais detalhe o DOLFIN

\section{PyAMG}
\label{sc:pyamg}
The obtained system of equations from the discretization with help of the \emph{FEniCS/DOLFIN} library was solved in this work using an Algebraic Multigrid (AMG) approach utilizing the \emph{PyAMG} package \citep{Bell2008}, which allows one to access a number of multigrid algorithms already implemented through a \emph{Python} interface. The most CPU-intensive routines of this package are implemented and compiled in \emph{C++} for performance purposes, but the user does not have to interface with this lower-level code.
% Explicar com MUITO mais detalhe o PyAMG