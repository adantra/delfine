\chapter{Introdu��o}
\label{ch:introduction}



\section{Motiva��o}
\label{sc:motivation}

Dentro do contexto atual de busca constante por um diferencial
competitivo, tem havido uma procura, tanto a n�vel acad�mico como
empresarial, de encontrar modos para maximizar a produtividade nas
mais diversas atividades. 

\citep{Savitch2004} \citep{Sorensen2001} \cite{Sorensen2001}
\citep{Mavriplis1998} \citep{Venkatakrishnan1994}
\citep{Bell2008} \citep{Wells2008} \citep{Li2004}
\citep{Langtangen2010} \citep{Martelli2006} \citep{Peaceman1977} \citep{Aziz1979}
\citep{Marle1981} \citep{Ewing1983}

A id�ia central do Multigrid consiste na escolha de um m�todo de
solu��o (suavizador) adequado para amortecer os erros associados �s
altas frequ�ncias, enquanto que os erros associados �s baixas frequ�ncias
s�o amortecidos atrav�s de malhas grosseiras, onde estes se manifestam
como frequ�ncias altas.

\section{Hist�rico do Multigrid}
\section{Hist�rico da Simula��o de Reservat�rios de Petr�leo}
\section{Objetivos}



